\documentclass[10pt,a4paper,final]{article}
\usepackage[latin1]{inputenc}
\usepackage{amsmath}
\usepackage{amsfonts}
\usepackage{amssymb}
\usepackage{graphicx}
\setlength{\topmargin}{-.5in}
\setlength{\textheight}{9in}
\setlength{\oddsidemargin}{.125in}
\setlength{\textwidth}{6.25in}
\author{Dennis Ideler}
\title{MATH 2P71: Intro to Combinatorics\\Assignment 3: Binomial Coefficients}
\begin{document}
\maketitle

\begin{enumerate}
\item % Q1
In how many ways can you distribute $n$ pennies to $k$ children
if each child is supposed to get at least 5?\\
\\
LPV 3.4.1: The number of ways to distribute $n$ identical pennies to $k$ children so that each child gets at least
one is $\binom{n-1k+k-1}{k-1} = \binom{n-1}{k-1}$.\\
LPV 3.4.2: The number of ways to distribute $n$ identical pennies to $k$ children is $\binom{n+k-1}{k-1}$.\\
\\
If each child is supposed to get at least 5 pennies, then we can give each child 5 pennies
first (i.e. a total of $5k$ pennies) to get rid of that constraint.
Now we have to count how many ways we can distribute the remaining $n-5k$ pennies.
Using theorem 3.4.2, we get $\binom{n-5k+k-1}{k-1} = \binom{n-4k-1}{k-1}$ ways.

\item % Q2
In a city with a regular ``chessboard'' street plan, the North-South streets are called 1st Street, 2nd
Street, $\cdots$, 20th Street, and the East-West streets are called 1st Avenue, 2nd Avenue, $\cdots$,
10th Avenue. What is the minimum number of blocks you have to walk to get from the corner of 1st Street
and 1st Avenue to the corner of 20th Street and 10th Avenue? In how many ways can you get there
walking this minimum number of blocks?\\
\\
This is Manhattan distance (also known as rectilinear distance, taxi cab metric, or city block distance)
and can be calculated with $\left|x_1-x_2\right| + \left|y_1-y_2\right|$.
Our starting point is $(1,1)$ and our ending point is $(20,10)$.
The minimum number of blocks needed is $\left|1-20\right| + \left|1-10\right| = 28$.\\
\\
The number of ways to travel this distance can be expressed as a binomial coefficient.
What are the number of $k$-subsets that can be formed from an $n$-set?
Our $n$ is the number of minimum blocks required, $n = 28$.
There are $\binom{28}{19} = \binom{28}{9}$ ways to choose 19 or 9 blocks out of 28 blocks.

\item % Q3
Provide a combinatorial argument to prove
\begin{equation*}
\sum_{k=0}^n \binom{n}{k} \binom{k}{m} = \binom{n}{m} 2^{n-m} % \sum\limits_{k=0}^n also works
\end{equation*}
Like any other combinatorial proofs, prove that the two different ways of counting are equivalent.\\

Let $A$ be $m$-subsets from $k$-subsets.
Let $B$ be $k$-subsets from the $n$-set.

On the LHS we have
\begin{equation*}
\sum_{k=0}^n \binom{n}{k} \binom{k}{m} = \binom{n}{0}\binom{0}{m} + \binom{n}{1}\binom{1}{m} +
\cdots + \binom{n}{n-1}\binom{n-1}{m} + \binom{n}{n}\binom{n}{m}
\end{equation*}

Each $\binom{n}{k} \binom{k}{m}$ counts the number of ways to choose $k$-subsets (B) from an $n$-set
and then choose $m$-subsets (A) from those $k$-subsets.

On the RHS we have
\begin{equation*}
\binom{n}{m} 2^{n-m}
\end{equation*}

We first choose an $m$-subset (A) from an $n$-set. There are $n \choose m$ ways to do this.
Then we choose elements from B from subsets of the relative complement of $B$ in $A$
($= A \setminus B = A - B$). There are $2^{|A-B|} = 2^{n-m}$ such subsets.\\

On both sides we have counted the number of pairs of subsets $(A,B)$ or in other words,
the number of pairs of $m$-subsets from a $k$-set and $k$-subsets from an $n$-set.

\item % Q4
Provide a combinatorial argument to prove
\begin{eqnarray*}
1 + \binom{n}{1}2 + \binom{n}{2}4 + \cdots + \binom{n}{n-1}2^{n-1} + \binom{n}{n}2^n = 3^n\\
=\binom{n}{0}2^0 + \binom{n}{1}2^1 + \binom{n}{2}2^2 + \cdots + \binom{n}{n-1}2^{n-1} + \binom{n}{n}2^n
= 3^n
\end{eqnarray*}

For example, on both the LHS and RHS we are counting the number of trinary strings of length $n$
(i.e. all strings over $\{0,1,2\}$ with length $n$).\\

RHS: Each digit in the string has 3 options (0, 1, or 2), and there are $n$ digits.
So there are $3^n$ trinary strings of length $n$.

LHS: Count the number of trinary strings of length $n$ by splitting them into partitions and
summing all partitions. Partition the strings based on the number of 0's, 1's, or 2's in the string
(they all yield the same result) into disjoint groups.
For example, if we partition based on number of 0's in the strings,
then we would have groups for strings with zero 0's, one 0, two 0's, three 0's, ... up to $n$ 0's.\\
\\
For example, if $n=3$ and we partition by number of 0's in a string.
\begin{eqnarray*}
\binom{3}{0}2^0 + \binom{3}{1}2^1 + \binom{3}{2}2^2 + \binom{3}{3}2^3\\
&=& (1 \cdot 1) + (3 \cdot 2) + (3 \cdot 4) + (1 \cdot 8)\\
&=& 1 + 6 + 12 + 8\\
&=& 27 \mbox{ trinary strings of length 3}\\
&=& 3^3
\end{eqnarray*}
Then there are 8 strings with zero 0's, 12 strings with one 0, 6 strings with two 0's, and 1 string
with three 0's (similar if we were to partition based off 1's or 2's).
Together, those are all the 27 possible trinary strings of length 3.

\end{enumerate}
\end{document}
