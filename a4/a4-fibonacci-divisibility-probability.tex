\documentclass[10pt,a4paper,final]{article}
\usepackage[latin1]{inputenc}
\usepackage{amsmath}
\usepackage{amsfonts}
\usepackage{amssymb}
\usepackage{graphicx}
\setlength{\topmargin}{-.5in}
\setlength{\textheight}{9in}
\setlength{\oddsidemargin}{.125in}
\setlength{\textwidth}{6.25in}
\author{Dennis Ideler}
\title{MATH 2P71: Intro to Combinatorics\\Assignment 4: Fibonacci and other stories}
\begin{document}
\maketitle

\begin{enumerate}
\item % Q1
Prove the following identities using simultaneous induction:
\begin{eqnarray}
\binom{n}{0}F_0 + \binom{n}{1}F_1 + \binom{n}{2}F_2 + \cdots + \binom{n}{n}F_n = F_{2n}\\
\binom{n}{0}F_1 + \binom{n}{1}F_2 + \binom{n}{2}F_3 + \cdots + \binom{n}{n}F_{n+1} = F_{2n+1}
\end{eqnarray}
\underline{Base case:}\\
For $n=1$
\begin{eqnarray}
\setcounter{equation}{1}
\binom{1}{0}F_0 + \binom{1}{1}F_1 = F_2 &\Rightarrow& (1)0 + (1)1 = 0+1 = 1 \: \checkmark\\
\binom{1}{0}F_1 + \binom{1}{1}F_2 = F_3 &\Rightarrow& (1)1 + (1)1 = 1+1 = 2 \: \checkmark
\end{eqnarray}
\underline{Simultaneous Induction:}\\
Assume both equalities true for $n$, then prove for one and use it and the induction hypothesis to prove
the other equality (i.e. use equation 2 in the proof of equation 1, and vice-versa).\\
\\
Equation 4.3 (pg 69 in our textbook) tells us that $F_{2n} = F_{n+1}F_n + F_nF_{n-1} = F_{2n-1} + F_{2n-2}$.\\
Thus $F_{2n+1} = F_{2n} + F_{2n-1}$, and from that we can derive $F_{2n} = F_{2n+1} - F_{2n-1}$.\\
\\
For equation 4.1 we use $F_{2n} = F_{2n+1} - F_{2n-1}$
\begin{eqnarray*}
&\binom{n}{0}F_0 + \binom{n}{1}F_1 + \binom{n}{2}F_2 + \cdots + \binom{n}{n}F_n \\ &=& \\
&\binom{n}{0}F_1 + \binom{n}{1}F_2 + \binom{n}{2}F_3 + \cdots + \binom{n}{n}F_{n+1} \\ &-& \\
&\binom{n-1}{0}F_0 + \binom{n-1}{1}F_1 + \binom{n-1}{2}F_2 + \cdots + \binom{n-1}{n-1}F_{n-1}
\end{eqnarray*}
For equation 4.2 we use $F_{2n+1} = F_{2n} + F_{2n-1}$
\begin{eqnarray*}
&\binom{n}{0}F_1 + \binom{n}{1}F_2 + \binom{n}{2}F_3 + \cdots + \binom{n}{n}F_{n+1} \\ &=& \\
&\binom{n}{0}F_0 + \binom{n}{1}F_1 + \binom{n}{2}F_2 + \cdots + \binom{n}{n}F_n \\ &+& \\
&\binom{n-1}{0}F_0 + \binom{n-1}{1}F_1 + \binom{n-1}{2}F_2 + \cdots + \binom{n-1}{n-1}F_{n-1}
\end{eqnarray*}

\item % Q2
We flip a coin $n$ times ($n \geq 1$). For which values of $n$ are the following pairs of events independent?
\begin{enumerate}
\item The first coin flip was heads; the number of heads was even. \\
\\
Two events are independent if information about one event (whether or not it occurred)
does not influence the probability of the other. Formally, two events $A$ and $B$
are independent if $P(A \cap B) = P(A)P(B)$.\\
Intuition tells me that the first flip being heads, likely has some influence on the number of heads,
but not on the number of heads being even/odd.\\
\\
Let $A = $ first flip was heads, and $B = $ number of heads was even.\\
$P(A) = \frac{1}{2}$ because a single coin flip has sample space $\{H,T\}$, both outcomes equally possible. \\
$P(B) = \frac{1}{2}$ because $\binom{n}{0} - \binom{n}{1} + \binom{n}{2} - \binom{n}{3} + \cdots \binom{n}{k}= 0$, where $n$ is the number of flips and $k$ is the number of heads.
Which can be rewritten as $\binom{n}{0} + \binom{n}{2} + \cdots = \binom{n}{1} + \binom{n}{3} + \cdots$
to indicate that the cardinality of $B$ (the number of ways to get even heads, LHS) is equal to the
cardinality of $\bar{B}$ (the number of ways to get odd heads, RHS).\\
\\
Event $A \cap B = $ set of outcomes such that the first coin was heads \emph{and} an odd number of
heads for the remaining $n-1$ flips (so the total number of heads is even).\\
$P(A \cap B) = \frac{1}{2} \times \frac{1}{2} = \frac{1}{4}$\\
$P(A)P(B) = \frac{1}{2} \times \frac{1}{2} = \frac{1}{4}$\\
$P(A \cap B) = P(A)P(B) \therefore$ independent for all $n \geq 2$.

\item The first coin flip was heads; the number of heads was more than the number of tails. \\
\\
Intuition tells me that if the first flip is heads, then there will generally be more heads than tails
(i.e. there is an influence).\\
Let $A = $ first flip was heads, and $B = $ number of heads was more than number of tails.\\
$P(A) = \frac{1}{2}$\\
$P(B) = \cdots$ ?\\
$n$ is not fixed, so determining the probability of $B$ is difficult.\\
Instead, we prove by contradiction.\\
The number of ways to have more heads than tails is
\begin{equation*}
|B| = \sum_{k \geq \lfloor \frac{n}{2} \rfloor + 1}^n \binom{n}{k}
\end{equation*}
where $n$ is the number of flips, and $k$ is the number of heads.
For example, $n = 6$ coin flips, then $\binom{6}{4} + \binom{6}{5} + \binom{6}{6}$ ways to have more
heads than tails.\\
\\
Event $A \cap B = $ set of outcomes such that first coin was heads \emph{and} number of heads
was more than number of tails.\\
\begin{equation*}
|A \cap B| = \sum_{k \geq \lfloor \frac{n}{2} \rfloor + 1}^n \binom{n}{k-1}
\end{equation*}
$k-1$ heads because first flip was heads.\\
\\
Consider $\bar{A} = $ first flip was tails. $P(\bar{A}) = \frac{1}{2}$.\\
Assume $A$ and $B$ are independent, then the pair $\bar{A}$ and $B$ must also be.\\
\\
Event $\bar{A} \cap B = $ set of outcomes such that first coin was tails \emph{and} number of heads
was more than number of tails.\\
\begin{equation*}
|\bar{A} \cap B| = \sum_{k \geq \lfloor \frac{n}{2} \rfloor + 1}^n \binom{n}{k}
\end{equation*}
$|A \cap B| \neq |\bar{A} \cap B|$. Contradiction, there is a different number of ways to do
a similar pair of events, which means there is some influence caused by the first event.
$\therefore$ dependent for all $n \geq 1$.

\item The number of heads was even; the number of heads was more than the number of tails. \\
\\
Let $A = $ number of heads was even, and $B =$ number of heads was more than the number of tails.
Similar to 2c, it's too difficult to find probability $P(B)$ so we prove by contradiction. \\
\\
Event $A \cap B =$ set of outcomes such that there is an even number of heads \emph{and}
there are more heads than tails. \\
\\
Let $\bar{A} =$ number of heads was odd.
Assume $A$ and $B$ are independent, then so must be $\bar{A}$ and $B$
(i.e. $|A \cap B| = |\bar{A} \cap B|$).\\

As in 2a
\begin{equation*}
|A| = \binom{n}{0} + \binom{n}{2} + \cdots
\end{equation*}
As in 2b
\begin{equation*}
|B| = \sum_{k \geq \lfloor \frac{n}{2} \rfloor + 1}^n \binom{n}{k}
\end{equation*}
So it follows that
\begin{eqnarray*}
|A \cap B| = \sum_{k \geq \lfloor \frac{n}{2} \rfloor + 1}^n \binom{n}{k} \: \mbox{,where $k$ is even}\\
|\bar{A} \cap B| = \sum_{k \geq \lfloor \frac{n}{2} \rfloor + 1}^n \binom{n}{k} \: \mbox{,where $k$ is odd}
\end{eqnarray*}
If $k$ is even
\begin{eqnarray*}
|A \cap B| = \binom{n}{k} + \binom{n}{k+2} + \cdots \\
|\bar{A} \cap B| = \binom{n}{k+1} +\binom{n}{k+3} + \cdots
\end{eqnarray*}
And so it follows that $|A \cap B| \geq |\bar{A} \cap B|$.\\
For example
\begin{eqnarray*}
n = 1, & &\binom{1}{0} = \binom{1}{1}, A \cap B = \mbox{false}\\
n = 2, & &\binom{2}{0} + \binom{2}{2} > \binom{2}{1}, A \cap B = \mbox{true}\\
n = 3, & &\binom{3}{0} + \binom{3}{2} = \binom{3}{1} + \binom{3}{3}, A \cap B = \mbox{false}\\
n = 4, & &\binom{4}{0} + \binom{4}{2} + \binom{4}{4} > \binom{4}{1} + \binom{4}{3}, A \cap B = \mbox{true}\\
n = 5, & &\binom{5}{0} + \binom{5}{2} + \binom{5}{4} = \binom{5}{1} + \binom{5}{3} + \binom{5}{5}, A \cap B = \mbox{false}\\
\vdots
\end{eqnarray*}
$\therefore |A \cap B| \neq |\bar{A} \cap B|$, independent for all $n \geq 1$.
\end{enumerate}

\item % Q3
Prove that every prime larger than 3 gives a remainder of 1 or 5 if divided by 6. \\
\\
$b = qa + r$, if $0 \leq r < a$, where $r$ is remainder and $q$ is quotient of $b/a$. \\
\\
Check if true for some primes greater than 3.\\
\begin{eqnarray*}
5 | 6 &\Rightarrow& 5 = (0)(6) + 5 \mbox{, where $q = 0$, $a = 6$, $r = 5$}\\
7 | 6 &\Rightarrow& 7 = (1)(6) + 1\\
11 | 6 &\Rightarrow& 11 = (1)(6) + 5\\
13 | 6 &\Rightarrow& 13 = (2)(6) + 1
\end{eqnarray*}
True so far. Prove for \emph{all} primes greater than 3.\\
$P \subseteq N^0$, and $N$ can be split into even numbers and odd numbers.\\
If $b$ is even, $b|6 \Rightarrow r = \{0,2,4\}$ because $b$ mod $6 = \{0,2,4\}$ with even $b$.\\
If $b$ is odd, $b|6 \Rightarrow r = \{1,3,5\}$ because $b$ mod $6 = \{1,3,5\}$ with odd $b$.\\
Thus for any $n,\: n|6 \Rightarrow r =\{0,1,2,3,4,5\}$.\\
\\
But no even numbers in set of primes (they are multiples of 2), so $\{0,2,4\}$ is excluded from $r$.
We start at primes larger than 3, so we know 3 is not in $r$ because no multiples of 3 are in 
our defined set of primes.\\
$\therefore$ we are left with $r = \{1,5\}$ which proves the original statement.

\item % Q4
We are given $n+1$ numbers from the set $S = \{1, 2, \cdots, 2n\}$.
Prove that there are two numbers among them such that one divides the other. \\
\\
Each integer (in $S$) can be written as an odd multiple of a power of 2.\\
For example
\begin{eqnarray*}
1 = 1 \times 2^0 \\
2 = 1 \times 2^1 \\
3 = 3 \times 2^0 \\
4 = 1 \times 2^2 \\
5 = 5 \times 2^0 \\
\vdots
\end{eqnarray*}
Can write this in the general form: $a2^b$, where $a$ is an odd number from $S$
(i.e. $a = \{1,3,5,\cdots,2n-1\}$, and $b \geq 0$. Thus there are $n$ odd numbers for $a$.\\
\\
From the examples above we see that numbers with the same ``box'' (i.e. same value for $a$)
are divisible by another. We can remove the powers of 2 from each integer to focus on the boxes
and apply the Pigeonhole Principle (PP).\\
\\
PP states that for $n$ items and $m$ boxes, at least one box will contain $\lceil \frac{n}{m} \rceil$
items.\\
In this case, there are $n+1$ numbers (items) and $n$ values for $a$ (boxes)
$\Rightarrow \lceil \frac{n+1}{n} \rceil = 2$ for $n \geq 1$.
So at least one box (odd number in $S$) is being used twice in the representation $a2^b$.\\
$\therefore$ there are two numbers among them such that one divides the other.\\
\\
Alternatively (but probably not as good a proof), it was stated that $S$ contains $\{1,2\}$, and $1|2$ 
so there are two numbers such that one divides the other.
\end{enumerate}
\end{document}
